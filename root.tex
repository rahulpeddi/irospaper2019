%%%%%%%%%%%%%%%%%%%%%%%%%%%%%%%%%%%%%%%%%%%%%%%%%%%%%%%%%%%%%%%%%%%%%%%%%%%%%%%%
%2345678901234567890123456789012345678901234567890123456789012345678901234567890
%        1         2         3         4         5         6         7         8

\documentclass[letterpaper, 10 pt, conference]{ieeeconf}  % Comment this line out if you need a4paper

%\documentclass[a4paper, 10pt, conference]{ieeeconf}      % Use this line for a4 paper

\IEEEoverridecommandlockouts                              % This command is only needed if 
                                                          % you want to use the \thanks command

\overrideIEEEmargins                                      % Needed to meet printer requirements.

%In case you encounter the following error:
%Error 1010 The PDF file may be corrupt (unable to open PDF file) OR
%Error 1000 An error occurred while parsing a contents stream. Unable to analyze the PDF file.
%This is a known problem with pdfLaTeX conversion filter. The file cannot be opened with acrobat reader
%Please use one of the alternatives below to circumvent this error by uncommenting one or the other
%\pdfobjcompresslevel=0
%\pdfminorversion=4

% See the \addtolength command later in the file to balance the column lengths
% on the last page of the document

% The following packages can be found on http:\\www.ctan.org
%\usepackage{graphics} % for pdf, bitmapped graphics files
%\usepackage{epsfig} % for postscript graphics files
%\usepackage{mathptmx} % assumes new font selection scheme installed
%\usepackage{times} % assumes new font selection scheme installed
%\usepackage{amsmath} % assumes amsmath package installed
%\usepackage{amssymb}  % assumes amsmath package installed
\usepackage{graphicx}
\usepackage{epstopdf}
\usepackage{amsmath}
\usepackage{amssymb}
%\usepackage{subfigure}
\usepackage{multirow}
\usepackage{pbox}
\usepackage{algorithm}
\usepackage{algorithmic}
%\usepackage{algpseudocode}
\usepackage{bm}
\usepackage{url}
\newcommand\NB[1]{$\spadesuit$\footnote{NB: #1}}
\newcommand\RP[1]{$\clubsuit$\footnote{RP: #1}}

\newcommand{\R}{\mathbb{R}}
\newcommand{\Z}{\mathbb{Z}}
\newcommand{\D}{\mathbb{D}}
\newcommand{\N}{\mathbb{N}}

\newtheorem{problem}{Problem}
\newtheorem{lemma}{Lemma}

\DeclareMathOperator*{\argmin}{arg\,min}
\DeclareMathOperator*{\argmax}{arg\,max}

\title{\LARGE \bf
Teleop Command/Imitation Learning/Regression (in progress)
}


\author{Rahul Peddi and Nicola Bezzo%
\thanks{Rahul Peddi and Nicola Bezzo are with the Department of Systems and Information Engineering and the Charles L. Brown Department of Electrical and Computer Engineering, University of Virginia, Charlottesville, VA 22904, USA. Email: {\tt \{rp3cy, nb6be\}@virginia.edu}}}


\begin{document}



\maketitle
\thispagestyle{empty}
\pagestyle{empty}


%%%%%%%%%%%%%%%%%%%%%%%%%%%%%%%%%%%%%%%%%%%%%%%%%%%%%%%%%%%%%%%%%%%%%%%%%%%%%%%%
\begin{abstract}


\end{abstract}


%%%%%%%%%%%%%%%%%%%%%%%%%%%%%%%%%%%%%%%%%%%%%%%%%%%%%%%%%%%%%%%%%%%%%%%%%%%%%%%%
\section{Introduction}

\section{Related Work}


\section{Problem Formulation}
In this work, we are interested in finding a policy that enables autonomous UAV navigation over a user-defined trajectory.

Formally, the problem we investigate in this work can be stated as:

\textbf{Problem 1: \textit{Demonstration-based autonomous control generation}}: A UAV has the objective to navigate over a certain trajectory. Trajectories are defined by a series of goal locations and times at which the goals are reached. Given $m$ flight demonstrations, find a policy to
\begin{enumerate}
    \item process and analyze the data in the training sets.%, where the inputs are distance travelled $d_i$ and time taken to travel that distance, $t_i$, where $i = 1,\ldots,m$ and $m$ is the number of trials.
    \item identify boundaries for user trajectories such that $d_{\min} \leq d_u \leq d_{\max}$ and $t_{\min} \leq t_u \leq t_{\max}$, where $d_u$ and $t_u$ denote user-set targets, or possibly waypoints within a trajectory, for the UAV.
    \item generate autonomous commands, $\mathbf{x}_a$.
    \item correct autonomous commands during run-time $\mathbf{x}_c$, subject to a constraint that UAV should stay within a certain distance of the user-set trajectory:
    \begin{equation}
        ||\mathbf{p}(t)-\mathbf{p}_r(t)|| \leq \delta,~\forall t \in [0,T]
    \end{equation}
    where $\mathbf{p}_r(t)$ is the reference position at time $t$, where $T$ represents an overall time horizon for the trajectory, and $\delta$ is a threshold for allowable deviation.
    
\end{enumerate}


\section{System Dynamics}
The type of UAV in question is modeled using a $12^{\text{th}}$ order state vector. 


\section{Approach}
In our approach, we leverage the data from multiple demonstrations to build a regression model that enables identification of an appropriate integral, $g_u$, and average velocity, $\bar{x}_u$ given a user-set distance, $d_u$, and time, $t_u$. With the appropriate integral and average velocity, the demonstrated command string with the closest integral, $g_i$, to the fitted integral, $g_u$, is resized to match the user-set time, $t_u$ and fitted average velocity, $\bar{x}_u$, to obtain the autonomous command string $\mathbf{x}_a$. This command is sent to the physical UAV for implementation, where the on-board computer collects data regarding the error between actual position, $\mathbf{p}(t)$, and the expected position, $\mathbf{p}_r(t)$ is determined based on the trajectory. This error is reduced by using a proportional controller on the input $\mathbf{x}_a$ to obtain the adjusted input commands, $\mathbf{x}_c$.


    *Block Diagram*

\subsection{Regression Based Training and Evaluation}
In order to build the appropriate policy for autonomous command generation, we perform offline training on data collected over $m$ human-piloted trials. Because the goal indicated in our problem statement is to be able to track a certain trajectory, the inputs of the training phase are $\{d_i,t_i,\mathbf{x_i}\}$, where $i=1,\ldots,m$, $d_i$ is the distance travelled in each trial, $t_i$ is the length of the trial, and $\mathbf{x_i}$ is the string of teleoperation commands given the by the pilot. In addition to the inputs given by the pilot, we are also interested in two entities \begin{itemize}
    \item The integral of the string of teleoperation commands, $g_i = \int_0^{t_i}\mathbf{x_i}$
    \item The steady-state teleoperation command, $\bar{x}_i \in \mathbf{x_i}$
\end{itemize}
The steady-state command and integral are necessary portions of the analysis because of the directly proportional relationship between teleop commands and system velocity. The steady-state command, in this case, provides information about the pilot's average in-flight velocity, and is obtained by taking the mean of all entries in $\mathbf{x_i}$. The integral, meanwhile, describes the area under the string of commands. Because the teleop commands are proportional to velocity, this area value gives important information about the distance traveled, as position is the integral of velocity over time.

The offline training data is applied to a thin-plate spline surface fit to describe the relationship between training inputs and integral and average velocity. The general form of a thin-plate spline equation is 
\begin{equation}
    f(x,y) = a_1 + a_xx + a_yy + \sum_{i=1}^mw_iU(||(x_i,y_i)-(x,y)||)
\end{equation}

where $a_1,a_x,\text{and}~a_y$ are scalar coefficients, $w_i$ is a coefficient that corresponds to each specific trial, subject to the following condition: \begin{equation}
\sum_{i=1}^mw_i=\sum_{i=1}^mw_ix_i=\sum_{i=1}^mw_iy_i=0
\end{equation}
and the function $U$ is of the form
\begin{equation}
  U(r) = r^2\log{r} 
\end{equation}

Given the corresponding $z_i$ for each $(x_i,y_i)$ pair, we are able to solve the following linear system to obtain the coefficients $w_i,\ldots,w_m$ and $a_1,a_x,a_y$,

\begin{equation}
    \begin{bmatrix}
    K&P\\
    P^T& \mathbf{0}
    \end{bmatrix}
    \begin{bmatrix}
    \mathbf{w}\\
    \mathbf{a}
    \end{bmatrix} = 
    \begin{bmatrix}
    \mathbf{z}\\
    \mathbf{o}
    \end{bmatrix}
\end{equation}

where $K_{ij} = U(||(x_i,y_i)-(x_j,y_j)||)$, $P_i* = (1,x_i,y_i)$, $\mathbf{0}  \in \mathbb{R}^{3\times3}$ is a matrix of zeros, $\mathbf{o} \in \mathbb{R}^{3\times1}$ is a column vector of zeros, $\mathbf{w} \in \mathbb{R}^{m\times1}$ and $\mathbf{z} \in \mathbb{R}^{m\times1}$ are formed from $w_i$ and $z_i$, respectively, and $\mathbf{a}$ is the column vector with elements $a_1,a_x,a_y$.

Given the general framework for performing the thin-plate spline, we develop two separate relationships for our specific application: $g_i = f(d_i,t_i)$ and $\bar{x}_i = h(d_i,t_i)$
With the functions we have obtained, we are able to find an estimated integral and steady-state command, $g_u$ and $\bar{x}_u$, for any given desired distance and time,

\begin{equation} \label{eq:integralfit}
g_u = f(d_u,t_u)
\end{equation}
\begin{equation} \label{eq:ssvelfit}
\bar{x}_u = h(d_u,t_u)
\end{equation}

where $d_u$ and $t_u$ are the user-set desired distance and time, respectively.

While a thin-plate spline is continuous, and a result can be obtained with any combination of distance and time as an input pair, the accuracy of the results of any pair $(d_u,t_u)$ can suffer as the distance between evaluation points and training points increases. In order to quantify this, we leverage the standard error of the estimate, which is a statistic used to measure the accuracy of predictions given a certain type of regression with known values:
\begin{equation} \label{eq:stderr}
    \sigma_{est} \approx \frac{s}{\sqrt{m}}
\end{equation}

where $s$ is the sample standard deviation of all of the points in the training set and $m$ is the number of training samples. In \eqref{eq:stderr}, an increased sample standard deviation results in . The standard error, $\sigma_{est}$, is then used to set the bounds for each training data point, for example:
\begin{align}
    \beta_{\min} = \hat{\beta} - \sigma_{est} \nonumber \\
    \beta_{\max} = \hat{\beta} + \sigma_{est}
\end{align}
where $\beta_{\min}$ and $\beta_{\max}$ are the lower and upper bounds for parameter $\hat{\beta}$.

In our case, there are two parameters we are primarily concerned about for prediction; distance and time. As a result, we perform this calculation twice over, treating each of the two parameters as statistically independent, to obtain a generic interval for each point in the training set, denoted $[d_{i\min},d_{i\max}]$ and $[t_{i\min},t_{i\max}]$, where $i=1,\ldots,m$. Because we are ultimately interested in using the distance and time values together as input pairs, we then obtain a maximum Euclidean distance from each of the training data points using:

\begin{equation}
    \Delta_i = \sqrt{(\sigma_{d_{est}})^2+(\sigma_{t_{est}})^2}
\end{equation}
Using the maximum distance for each point $\Delta_i$, we are able the generate intervals around each point, where the data is within the standard error of the estimate.


\subsection{Autonomous Behaviour Generation}

In order for the system to autonomously reach a user-set goal, $g_u$ and $\bar{x}_u$ are obtained using equations \eqref{eq:integralfit} and \eqref{eq:ssvelfit}, and the offline training samples are leveraged to generate a new string of commands.

From the training set of $m$ samples, we select the trial that has the closest integral, $g_i$, to the estimated integral $g_u$. This is done by forming an error vector, $\mathbf{e}\in\R^{1\times m}$, where each element is defined by
\begin{equation}
 e_i = \vert a_i-a_u \vert , ~i\in \{1,\ldots,m\}
\end{equation}
 The lowest error is then found and is paired with the appropriate pre-trained sample, $\mathbf{x}^*$,

\begin{equation}
\mathbf{x}^* = \mathbf{x}_i \in \mathbf{x}\vert e_i = \min_e(\mathbf{e})
\end{equation}

This optimal pre-trained sample is then adjusted to reflect the user-set time, $t_u$. This is done by performing bicubic interpolation to resize the vector $\mathbf{x}^*$. Bicubic interpolation is the chosen method for resizing, as it performs better than nearest-neighbor and bilinear interpolation methods, while only marginally increasing computational complexity. The general form of a bicubic interpolation equation is 
\begin{equation}
    p(x) = \sum_{i=0}^3a_ix^i
\end{equation}

where $x$ is an entry in vector $\mathbf{x}^*$ and $a$ represents the coefficients of the function at each point. Bicubic interpolation takes the weighted sum of the four nearest neighbors of each entry in the command vector in order to identify the intermediate points between each value in $\mathbf{x}^*$. After resizing, we obtain the time adjusted input vector $\mathbf{x}'$. 

The next step is to adjust the input vector such that the system reaches the user-set goal $d_u$. This is done by leveraging the average velocity information, that is, $\bar{x}_u$. Because distance is a function of average velocity and time, the scale time-adjusted vector $\mathbf{x}'$ is scaled such that its mean is equivalent to $\bar{x}_u$. We then obtain
\begin{equation}
\mathbf{x}_a = \mathbf{x}'\bigg(\frac{\bar{x}_u}{\bar{x}'}\bigg)
\end{equation}

The input $\mathbf{x}_a$ is then sent to the UAV to reach the goals set by the users.


\subsection{Online Adaptation of Generated Commands}

The commands generated in the previous section are generated and sent to the UAV prior to any testing or evaluation. Therefore, we propose a method to correct for any error that may occur during evaluation.

\subsection{Trajectory Decomposition}


\section{Experiments}


\section{Conclusions}

\section{Acknowledgement}


\addtolength{\textheight}{-12cm}   % This command serves to balance the column lengths
                                  % on the last page of the document manually. It shortens
                                  % the textheight of the last page by a suitable amount.
                                  % This command does not take effect until the next page
                                  % so it should come on the page before the last. Make
                                  % sure that you do not shorten the textheight too much.

%%%%%%%%%%%%%%%%%%%%%%%%%%%%%%%%%%%%%%%%%%%%%%%%%%%%%%%%%%%%%%%%%%%%%%%%%%%%%%%%

%References are important to the reader; therefore, each citation must be complete and correct. If at all possible, references should be commonly available publications.



\begin{thebibliography}{99}

\bibitem{c1} G. O. Young, �Synthetic structure of industrial plastics (Book style with paper title and editor),� 	in Plastics, 2nd ed. vol. 3, J. Peters, Ed.  New York: McGraw-Hill, 1964, pp. 15�64.
\bibitem{c2} W.-K. Chen, Linear Networks and Systems (Book style).	Belmont, CA: Wadsworth, 1993, pp. 123�135.
\bibitem{c3} H. Poor, An Introduction to Signal Detection and Estimation.   New York: Springer-Verlag, 1985, ch. 4.


\end{thebibliography}




\end{document}
